\documentclass{nzlaw}
\usepackage{float}
\usepackage{minted}

\addbibresource{sample.bib}
\lhead{}
\newcommand{\s}{\textbackslash}
\newcommand{\latex}{\LaTeX}
\begin{document}
\title{NZ Law Latex Style Instructions}
\author{Roman Klapaukh}
\date{\today}

\maketitle

\section{General Rules: Main Text}
\subsection{Prose}
\subsubsection{Language}

\paragraph{Foreign Languages}
Foreign language characters such as \"{e} can be typed directly into the  text. However, for those who do not know how to type them, the following table contains commands for generating common accentented letters. Examples use the letter a where possible for simplicity.

\begin{table}[H]
\centering
\begin{tabular}{|l|l|}
\hline
Accent & Command \\ \hline
\"{a} & \s''\{a\} \\
\`{a} & \s`\{a\} \\
\'{a} & \s'\{a\} \\
\^{a} & \s\^\{a\}\\
\~{a} & \s\textasciitilde\{a\}\\
\c{c} & {\s}c\{c\}\\
\k{a} & {\s}k\{a\}\\
\={a} & \s=\{a\}\\
\d{a} & {\s}d\{a\}\\
\r{a} & {\s}r\{a\}\\
\u{a} & {\s}v\{a\}\\
\hline
\end{tabular}
\caption{Accented Letters}
\end{table}

\paragraph{Punctuation}
Punctuation in \latex~is the same as in normal text. \latex~adds a slight amount of extra space following full stops. Careful authors using fullstops in the middle of text (which should not happen in general) can stop this happening by using a \textasciitilde~instead of a space.

\begin{tabular}{l l}
Original & Output \\
Hello. This is a Word. With. Dots. & Hello. This is a Word. With. Dots. \\

Hello. This is a Word.{\textasciitilde}With.{\textasciitilde}Dots. &Hello. This is a Word.~With.~Dots.
\end{tabular}

\paragraph{Italics and Emphasis}
The following table provides commands for emaphsis and such. While bold and underline are not expected to be used, they are provided in case.

\begin{table}[H]
\centering
\begin{tabular}{|l|l|l|}
\hline
Text Type & Command & Example \\ \hline
Italics & {\s}textit  & {\s}textit\{some text\}\\
Bold & {\s}textbf  & {\s}textbf\{some text\}\\
Underline & {\s}underline & {\s}underline\{some text\} \\
\hline
\end{tabular}
\caption{Text Formatting Commands}
\end{table}

\section{General Rules: Footnotes and the Citation of Sources}



\section{Cases}

\subsection{Principles for Citing Cases}

\subsubsection{Neutral Citations}
Neutral citations are written as follows
\begin{minted}{latex}
@case{astrazeneca,
title={AstraZeneca Ltd v Comerce Commission},
origdate={2009},
venue={NZSC},
number={92},
}
\end{minted}

to get \cite{astrazeneca}.

\subsection{Reported Cases}
If they have a neutral citation they contain the neutral citation, as well as whatever extra information is needed. Otherwise they just have whatever information is available

\begin{minted}{latex}
@case{z,
title = {Z v Dental Complaints Assessment Committee},
origdate = {2008},
court = {NZSC},
number = {55},
year ={2009},
volume = {1},
series = {NZLR},
pages = {1},
}
\end{minted}

\begin{minted}{latex}
@case{taylor,
title = {Taylor v New Zealand Poultry Board},
year ={1984},
volume = {1},
series = {NZLR},
pages = {394},
venue = {CA},
}
\end{minted}
\subsubsection{Common Names}

Common names are included as shorttitle.
\begin{minted}{latex}
@case{baigents,
title = {Simpson v Attorney-General},
year = {1994},
volume = {3},
series = {NZLR},
pages = {667},
venue = {CA},
shorttitle = {Baigent's case},
}
\end{minted}

\subsubsection{Year}

Round brackets for the year can be forced by including type = {round}.
\begin{minted}{latex}
@case{peterson,
title = {Peterson v Commissioner of Inland Revenue},
year = {2005},
volume = {22},
series = {NZTC},
pages = {\num{19098}},
venue = {PC},
type = {round},
}
\end{minted}


Old style ones where both round brackets and no court brackets are used are done with type = {old}.
\begin{minted}{latex}
@case{zohrab,
title = {Zohrab v Fuller},
year = {1884},
volume = {3},
series = {NZLR},
venue = {SC},
pages = {210},
type = {old},
}
\end{minted}

\subsubsection{Starting Page}

The pilcrow sign (\textparagraph) can be printed with {\s}textpilcrow.

\begin{minted}{latex}
@case{blacker,
title = {Blacker v National Australis Bank Ltd},
origdate = {2001},
year = {2001},
venue = {FCA},
number = {254},
volume = {23},
type = {round},
series = {ATPR},
pages = {\textparagraph41-- 817},
}
\end{minted}

\end{document}