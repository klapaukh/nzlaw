\documentclass{book}
\usepackage{float}
\usepackage[group-separator={,}]{siunitx} 
\usepackage[citestyle=nzlaw,hyperref=true,backend=biber]{biblatex}
\usepackage{hyperref}
\usepackage{fullpage}
\usepackage{fancyhdr}
\usepackage{fontspec}
\usepackage{titlesec}
\usepackage{xcolor}
\usepackage[draft]{ifdraft}
\ifdraft{
  \usepackage{fancyvrb}
  \DefineVerbatimEnvironment{minted}{Verbatim}{}
  \DefineVerbatimEnvironment{bib}{Verbatim}{}
}{
  \usepackage{minted} % syntax coloring. 
  \newminted[bib]{latex}{linenos=false,fontfamily=helvetica}
  \usemintedstyle{vim}
}

\setcounter{secnumdepth}{3}
\titleformat{\chapter}[hang]{\color{black}\bfseries\large}{\thechapter}{3em}{\MakeUppercase}[]

\titleformat{\section}[hang]{\bfseries\large}{\thesection}{3em}{\MakeUppercase}[{\color{black}\titlerule[0.5mm]}]

\titleformat{\subsection}[hang]{\bfseries\large}{\thesubsection}{3em}{}[{\color{gray}\titlerule[0.1mm]}]

\renewcommand{\thesubsubsection}{(\alph{subsubsection})}
\titleformat{\subsubsection}[hang]{\bfseries}{\thesubsubsection}{3em}{}[]

\addbibresource{sample.bib}

\newcommand{\s}{\textbackslash}
\newcommand{\latex}{\LaTeX}
\setlength{\parindent}{0em}
\setlength{\parskip}{2ex}


\fancyhf{}
\fancyfoot[OR]{New Zealand Law Latex Style Instructions \qquad | \qquad \thepage}
\fancyfoot[EL]{\thepage \qquad | \qquad New Zealand Law Latex Style Instructions}
\renewcommand{\headrulewidth}{0pt}
\renewcommand{\footrulewidth}{0pt}

\fancypagestyle{plain}{ %
\fancyfoot[OR]{New Zealand Law Latex Style Instructions \qquad | \qquad \thepage}
\fancyfoot[EL]{\thepage \qquad | \qquad New Zealand Law Latex Style Instructions}
\renewcommand{\headrulewidth}{0pt} % remove lines as well
\renewcommand{\footrulewidth}{0pt}
}


%\newenvironment{bib}{}{}

\begin{document}



\title{New Zealand Law Latex Style Instructions}
\author{Roman Klapaukh}
\date{\today}
\frontmatter
\pagestyle{fancy}

\maketitle


\tableofcontents

\newpage
\addcontentsline{toc}{chapter}{\protect\numberline{}Preface}
{\noindent\large\bfseries\MakeUppercase{Preface}\hrule \vspace{2ex}}

This is an attempt to provide an alternative to typesetting Microsoft Word. 
This is a package for a free typesetting sysetm known as \LaTeX. 
All software required to run use this is freely available and will work on 
most common operating systems including Windows, Mac OS, and Linux.

The purpose is to allow users to write their documents with no consideration
for formatting, and have the formatting be performed automatically by the computer.
As this is a small project this is impractical in general. However, care will be taken
to make this as true as possible. 

This manual documents the usage of the \LaTeX package which formats documents
according to the rules set out in the New Zealand Law Style Guide (2nd ed).
Unfortunately it is not yet complete and probably contains numerous errors.
In the case of missing functionality or incorrect output please contact the 
author. This document has been written to mimic the  structure and appearance
of the New Zealand Law Style Guide to simplify uptake for people already 
familiar with manually writing citations and formating.


\addcontentsline{toc}{chapter}{\protect\numberline{}General Principles}
{\noindent\large\bfseries\MakeUppercase{General Principles} \hrule\vspace{2ex}}
This guide is written on the principle that all the resources required should be freely available to all possible interested parties. 
There should be no restrictions irrespective of the user. 
This imposes no extra limitations in commercial use over use in educational contexts.

\addcontentsline{toc}{chapter}{\protect\numberline{}Errors and Omissions}
{\noindent\large\bfseries\MakeUppercase{Errors and Omissions}\hrule\vspace{2ex}}
This document is full of errors and omissions. 
Any feedback or corrections would be much appreciated. 
It is hoped that eventually this document will be mostly correct.


\mainmatter
\chapter{General Rules: Main Text}

Every document created using this package will have the extension \texttt{.tex} and
the following general layout:

\begin{bib}
\documentclass{nzlaw}

\addbibresource{bibliograph.bib}
\studentid{312345678}
\wordcount{1000}

\begin{document}

[Your document text here]

\end{document}
\end{bib}

The \texttt{{\s}documentclass} is an instruction on what formatting rules to apply to this
document. There are a large number of these classes that have been written. However,
in order to get formatting using the nzlaw class that must contain the word ```nzlaw''.

\texttt{{\s}addbibresource} contains the file that all of your citations are in. In 
\LaTeX documents every reading that is done is entered into the bilibography file.
These can then be referenced within the document as citations. This file must be
in the same directory as the tex file for the document. A large part of this 
document will specify the format of this bibliograph file. 

The commands \texttt{{\s}studentid} and \texttt{{\s}wordcount} are provided for the
convience of students submitting essays. The values in them should be replaced by the
document authors student id and essay word count. These will then be placed in the
document header. 

\texttt{{\s}begin\{document\}} and \texttt{{\s}end\{document\}} mark the start and
end of your text. The text must be written sequentially as you would in any other
circumstance. 

\section{Prose}
\subsection{Language}

\LaTeX has no ability to check your language for you. While it will provide formatting
it requires you to make sure you follow the appropriate style rules, as well as spelling
and grammar checking the document yourself. No amount of automation can as of yet replace
proof reading. 

\subsubsection{Foreign Languages}
Foreign language characters such as \"{e} can be typed directly into the text. 
However, for those who do not know how to type them, the following table contains 
commands for generating common accented letters. 
Examples use the letter a where possible for simplicity.

\begin{table}[H]
\centering
\begin{tabular}{|l|l|}
\hline
Accent & Command \\ \hline
\"{a} & \s''\{a\} \\
\`{a} & \s`\{a\} \\
\'{a} & \s'\{a\} \\
\^{a} & \s\^\{a\}\\
\~{a} & \s\textasciitilde\{a\}\\
\c{c} & {\s}c\{c\}\\
\k{a} & {\s}k\{a\}\\
\={a} & \s=\{a\}\\
\d{a} & {\s}d\{a\}\\
\r{a} & {\s}r\{a\}\\
\u{a} & {\s}v\{a\}\\
\hline
\end{tabular}
\caption{Accented Letters}
\end{table}

\subsubsection{Punctuation}
Punctuation in \latex~is the same as in normal text. 
By default this format adds a slight amount of extra space following full stops. 
This is known as English spacing (as opposed to French spacing), except that these
extra spaces should only occur after full stops at the end of a sentence. However,
\LaTeX  is unable to tell if full stop denotes the end of a sentence and so may 
insert these extra spaces incorrectly.  
Careful authors using fullstops in the middle of text (which should not happen in general) 
can stop this happening by using a {\s}@ after the fullstop. 
If a sentence ends on a capital letter it will not put in an exapanded space after the
 full stop as it assumes that the sentence isn't finished. 
To correct this put {\s}@ before the full stop. 
Note how this is the same as to shrink the space, except placed before the full stop.
If you do not like having extra spaces after the full stop, you can put {\s}frenchspacing 
at the begining of your document to make all spaces the same. 

Here are some example sentences to Illustrate this effect.

\begin{tabular}{l l}
Original & Output \\
Hello. This is a sentence. with. dots. & Hello. This is a sentence. with. dots. \\

Hello. This is a sentence.{\s}@ with.{\s}@ dots. &Hello. This is a sentence.\@ with.\@ dots.\\

HOWDY. How are you? & HOWDY. How are you?\\
HOWDY{\s}@. How are you? & HOWDY\@. How are you?\\
\end{tabular}

\subsubsection{Italics and Emphasis}
In order to make a section of text italicised, or emphasised in any other way,
the appropriate text should be put inside the appropriate instruction.
The following table provides commands for emaphsis and such. 
While bold and underline are not expected to be used, they are provided in case.

\begin{table}[H]
\centering
\begin{tabular}{|l|l|l|}
\hline
Text Type & Command & Example \\ \hline
Italics & {\s}textit  & {\s}textit\{some text\}\\
Bold & {\s}textbf  & {\s}textbf\{some text\}\\
Underline & {\s}underline & {\s}underline\{some text\} \\
\hline
\end{tabular}
\caption{Text Formatting Commands}
\end{table}

\chapter{General Rules: Footnotes and the Citation of Sources}

\section{Footnotes and author notes}
Footnotes can be inserted in \LaTeX in a similar fashion to italics. To footnote
text, you simple write {\s}footnote\{Footnote text\} at the place you wish to 
insert the footnote. The text inside the curly braces will be the text of the footnote.
Here is an example.\footnote{Footnote text}. As the law citation style uses footnote
citations footnotes and citations share a common numbering system.


\section{Citations}

Citations are slightly more complicated when using \LaTeX than when using word. 
This is because \LaTeX needs to keep track of all possible citations so they can
be used in your document when needed. In order to do this it uses a special
bibliography file.  

\chapter{Cases}

\section{Principles for Citing Cases}

\subsection{Neutral Citations}
Neutral citations are written as follows
\begin{bib}
@case{astrazeneca,
title={AstraZeneca Ltd v Comerce Commission},
origdate={2009},
court={NZSC},
number={92},
}
\end{bib}

to get \cite{astrazeneca}.

\section{Reported Cases}
If they have a neutral citation they contain the neutral citation, as well as whatever extra information is needed. Otherwise they just have whatever information is available

\begin{bib}
@case{z,
title = {Z v Dental Complaints Assessment Committee},
origdate = {2008},
court = {NZSC},
number = {55},
year ={2009},
volume = {1},
series = {NZLR},
pages = {1},
}
\end{bib}

\begin{bib}
@case{taylor,
title = {Taylor v New Zealand Poultry Board},
year ={1984},
volume = {1},
series = {NZLR},
pages = {394},
court = {CA},
}
\end{bib}
\subsection{Common Names}

Common names are included as shorttitle.
\begin{bib}
@case{baigents,
title = {Simpson v Attorney-General},
year = {1994},
volume = {3},
series = {NZLR},
pages = {667},
court = {CA},
shorttitle = {Baigent's case},
}
\end{bib}

\subsection{Year}

Round brackets for the year can be forced by including type = {round}.
\begin{bib}
@case{peterson,
title = {Peterson v Commissioner of Inland Revenue},
year = {2005},
volume = {22},
series = {NZTC},
pages = {\num{19098}},
court = {PC},
type = {round},
}
\end{bib}


Old style ones where both round brackets and no court brackets are used are done with type = {old}.
\begin{bib}
@case{zohrab,
title = {Zohrab v Fuller},
year = {1884},
volume = {3},
series = {NZLR},
court = {SC},
pages = {210},
type = {old},
}
\end{bib}

\subsection{Starting Page}

The pilcrow sign (\textparagraph) can be printed with {\s}textpilcrow.

\begin{bib}
@case{blacker,
title = {Blacker v National Australis Bank Ltd},
origdate = {2001},
year = {2001},
court = {FCA},
number = {254},
volume = {23},
type = {round},
series = {ATPR},
pages = {\textparagraph41-- 817},
}
\end{bib}

\subsection{Judge Identifier}

Judge identifiers can just be added to the pinpoint e.g. as {\s}footcite[\{[48]\} per Tipping J]\{mafart\} to get \cite[{[48]} per Tipping J]{mafart}.

\subsection{Case History}
aff'd and rev'd can be added in an extra set of [] from the pinpoint. The second case name is automatically suppressed if they are the same.  E.g. 


\noindent{\s}footcites\{foodstuffs\}[rev'd][]\{foodstuffs1\} gives
\cites{foodstuffs}[rev'd][]{foodstuffs1}


\section{Unreported Cases - Neutral Citation}



\section{Unreported Cases - File Number Citation (Supreme Court and Court of Appeal)}

\begin{bib}
@case{reekie,
title = {R v Reekie},
number = {CA339/03},
date = {2004-08-03},
type = {unreported},
}
\end{bib}


\section{Unreported Cases - File Number Citation (Including the High Court and District Court)}

\begin{bib}
@case{tuhou,
title = {R v Tuhou},
court = {HC},
location = {Napier},
number = {CRI-2007-020-2820},
date = {2008-09-11},
type = {unreported},
}
\end{bib}

\section{M\=aori Land Court and M\=aori Appellate Court Decisions}

\begin{bib}
@case{craig,
title = {Craig v Kira},
location = {Wainui 2F4D},
year = {2006},
file = {7 Whangarei Appellate MB 1},
note = {7 APWH 1},
type = {maori},
}
\end{bib}


\chapter{Legislation}

\section{statues}
\begin{bib}
@statute{gaming,
title = {Gaming Duties Act},
year = {1971},
}
\end{bib}
\begin{bib}
@statute{terror,
title = {Counter-Terrorism Act},
year = {2008},
jurisdiction = {UK},
}
\end{bib}
\subsection{Pre-1853 Ordinances}

\begin{bib}
@statute{distillation,
title = {Distillation Prohibition Ordinance},
year = {1841},
regnalyear = {4 Vict},
ordinanceno = {5},
}
\end{bib}


\section{Bills}

\begin{bib}
@bill{judicial,
title = {Judicial Matters Bill},
year = {2008},
number = {216-1},
}
\end{bib}

\subsection{Select Committee Reports, and Explanatory  Notes}
\begin{bib}
@bill{judicialex,
title = {Judicial Matters Bill},
year = {2008},
number = {216-1},
note = {explanatory note},
}
\end{bib}


\subsection{Supplementary Order Papers}
\begin{bib}
@bill{evidence,
title={Evidence Bill},
year = {2005},
number = {256-1},
sopnumber = {79},
sopyear={2006},
note = {explanatory note},
}
\end{bib}

\section{Regulations}

\begin{bib}
@regulations{costs,
title = {Costs in Criminal Cases Regulations},
year = {1987},
}
\end{bib}


\chapter{Other Official Sources}

\section{Parliamentary Materials}

\subsection{Hansard}

\begin{bib}
@hansard{hansard,
title= {NZPD},
date = {2005-04-06},
volume = {624},
}
\end{bib}

\subsection{Appendix to the Journals of the House of Representatives}

\begin{bib}
@appendix{palmer85,
author = {Geoffrey Palmer},
title = {A Bill of Rights for New Zealand: A White Paper},
date = {1984/1985},
volume = {I},
number = {A6},
}
\end{bib}

\subsection{Submissions to Select Committees}
\begin{bib}
@submission{library,
author = {New Zealand Law Librarian Group Inc},
title = {Submission to the Justice and Law Reform Committee on the Interpretation of Bill 1998},
}
\end{bib}

\subsection{Standing Orders}
\begin{bib}
@orders{2008,
year= {2008},
}
\end{bib}

\section{Government Publications}

\subsection{Cabinet Documents}
\begin{bib}
@cabinetdoc{conduct,
label = {Cabinet Office Circular},
title = {Conduct During Periods of Caretaker Government},
date = {1999-04-21},
number = {CO 99/5},
}
\end{bib}

\subsection{Cabinet Manual}
\begin{bib}
@cabinetman{cabman,
year = {2008}
}
\end{bib}

\subsection{New Zealand Gazetter}
\begin{bib}
@nzgazette{ad,
title = {Advertisement of Application to put Company into Liquidation by the Court},
number = {MO No 270/97},
date = {1997-08-21},
}
\end{bib}

\subsection{Law Commission Reports}
\begin{bib}
@nzlc{prosecution,
title = {The Prosecution of Offences},
number = {PP12},
year = {1990},
}
\end{bib}

\subsection{Other Reports}
\begin{bib}
@report{salmon09,
author = {Peter Salmon and Margaret Bazley and David Shand},
title  = {Royal Commission on Auckland Governance},
publisher = {Some publisher}
date = {2009-03},
}
\end{bib}
\begin{bib}
@report{price,
author = {PricewaterhouseCoopers},
title = {Commerce Commission Baseline Review},
note = {prepared for the Ministry of Economic Development},
date = {2008},
}
\end{bib}

\section{Requests Under the Official Information Act 1982}

Current Omitted. Can be added on request

\chapter{Secondary Materials}

\section{Texts}
\begin{bib}
@book{todd01,
editor={Stephen Todd and Gordon Ramsey},
title = {The Law of Torts in New Zealand},
edition = {3rd},
publisher = {Brookers},
location = {Wellington},
year = {2001},
}
\end{bib}
\begin{bib}
@book{butler05,
author = {Andrew Butler and Petra Butler},
title = {The New Zealand Bill of Rights Act: A Commentary},
edition = {1st},
publisher = {LexisNexis},
location = {Wellington},
year = {2005},
}
\end{bib}
\begin{bib}
@book{butler03,
editor = {Andrew Butler},
title = {Equity and Trusts in New Zealand},
publisher = {Brookers},
location = {Wellington},
date = {2003/2004},
}
\end{bib}


\section{Essays in Edited Books}
\begin{bib}
@inbook{cooke87,
author = {Robin Cooke},
title = {Tort and Contract},
editor = {PD Finn},
booktitle = {Essays on Contract},
publisher = {Law Book Company},
location = {Sydney},
year = {1987},
pages = {222},
}
\end{bib}

\section{Looseleaf Texts}
\begin{bib}
@book{robertson,
editor = {Bruce Robertson},
edition = {looseleaf},
title = {Adams on Criminal Law},
publisher = {Brookers},
}
\end{bib}
\begin{bib}
@book{gault,
editor = {Thomas Gault},
edition ={online looseleaf},
title = {Gault on Commercial Law},
publisher = {Brookers},
year = {accessed 15 June 2009},
}
\end{bib}

\section{Journal Articles}
\begin{bib}
@article{watts05,
author = {Peter Watts},
title = {Birk's Unjust Enrichment},
year = {2005},
volume = {121},
journaltitle = {LQR},
pages = {163},
}
\end{bib}
\begin{bib}
@article{palmer07,
author = {Jessica Palmer},
title={Dealing with the Emerging Popularity of Sham Trusts},
year = {2007},
journaltitle = {NZ Law Review},
pages = {81}
}
\end{bib}

\section{Legal Encylopaediae}
\begin{bib}
@reference{companies2,
title = {Halsbury's Laws of England},
edition = {2nd},
year = {1932},
volunm = {5},
label = {Companies},
}
\end{bib}
\begin{bib}
@reference{companies1,
title = {Halsbury's Laws of England},
edition = {1st},
year = {1932},
volume = {5},
label = {Companies},
}
\end{bib}
\begin{bib}
@reference{contract,
title = {Halsbury's Laws of England},
year = {1998},
volume = {5(1)},
label = {Contract},
note = {reissue}
}
\end{bib}

\section{Laws of New Zealand}
\begin{bib}
@reference{equity,
title = {Laws of New Zealand},
label = {Equity},
}
\end{bib}
\begin{bib}
@reference{equityonline,
title = {Laws of New Zealand},
label = {Equity},
edition = {online},
}
\end{bib}
\begin{bib}
@reference{parliment,
title = {Laws of New Zealand},
label = {Parliament},
edition = {online},
note = {Reissue 1},
}
\end{bib}

\section{Unpublished Papers}

\subsection{Theses and Research Papers}
\begin{bib}
@thesis{roberts08,
author = {Marcus Roberts},
title = {Reforming New Zealand's Legislative Council: A Study of Constitutional Change, 1891--1920},
type = {LLB (Hons) Dissertation},
institution = {University of Auckland},
year = {2008},
}
\end{bib}

\subsection{Seminars, and Papers Presented at Conferences}
\begin{bib}
@unpublished{epps04,
author = {Tracey Epps},
title = {Merchants in the Temple? The Implications of the GATS for Canada's Health Care System},
location = {National Health Law Conference, Toronto},
date = {2004-01},
type = {paper},
}
\end{bib}

\chapter{Other Sources}

\section{Internet Materials}
\begin{bib}
@online{knight09,
author = {Dean Knight},
title = {Parliment and the Bill of Rights --- a blas\'{e} attitude?},
year = {2009},
maintitle = {LAWS179 Elephants and the Law},
url = {www.laws179.co.nz},
}
\end{bib}

\section{Newspaper Articles}
\begin{bib}
@press{hosking09,
author = {Rob Hosking},
title = {Messy Allowance Law Finally Gets Clarity},
publisher = {The National Business Review},
location = {New Zealand},
date = {2009-07-17},
}
\end{bib}

\section{Interviews}

Omitted. Addition maybe on special request

\section{Press Releases}
\begin{bib}
@press{airnz,
author = {Air New Zealand},
title = {Lock-out Notice issued to EPMU},
type = {press release},
date = {2009-04-21},
}
\end{bib}

\section{Speeches}
\begin{bib}
@unpublished{elias08,
author = {Sian Elias},
authortype = {Chief Justice of New Zealand},
title = {First Peoples and Human Rights, a South Seas Perspective},
type = {speech},
location = {Ramo Lecture 2008, New Mexico School of Law, Albuquerque},
date = {2008-10-23},
}
\end{bib}

\section{Letters and Emails}
Omitted. May be added on request. Maybe

\end{document}
