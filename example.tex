\documentclass{nzlaw}

\addbibresource{sample.bib}


%opening
\studentid{123456789}
\wordcount{1334}

\begin{document}

\section{Introduction}
\label{sec:Introduction}

This is an example to how to use the nzlaw style format. 
There are commands for quoting: \shortquote{Hello all},
and way of writing the three different kinds of dashes -,--,---.
You can also cite papers with pinpoints.\footcite[99]{astrazeneca}  
Things like ibid citations are automatically taken care 
of.\footcite[99]{astrazeneca} 
There is a slight issue if you want to use square brackets ([]) in a 
pinpoint for a citations, where you have to do them this
way.\footcite[22 and {[5]}]{astrazeneca} 
Above n x ciations are also supported.\footcite{z}
Automatically.\footcite[22]{astrazeneca} 

With citations for legislation you should always use the short
form in the pinpoint.\footcite[s 2]{gaming}
When it is used as an ibid citation it will automatically be 
expanded.\footcite[s 5]{gaming}

\subsection{Sub headings}
We have a number of different kinds of headings available. 
The full list is section, subsection, subsubsection, paragraph
and subparagraph. Each is automatically formatted in the correct 
style. 

\subsubsection{Text emphasis}
Some times when writing text we wish to emphasise certain words.
This can be done with \emph{italics}, \textbf{bold} or \textsc{small
caps}. Underlining is possible, but shouldn't be done.

\paragraph{Quoting}

There are two kinds of quotes. Short quotes \shortquote{of under 30
words}. And long quotes:

\begin{longquote}
These require separate spacing, fontsizes and margins. Fortunately
this is already done.
\end{longquote}



\subparagraph{More citations}
Sometimes you want more complicated pinpoints. Fortunately this is
just the same as a normal pinpoint.\footcite[{[48]} per Tipping J]{mafart}
You may also wish to cite multiple cases in one citations, perhaps
even related ones.\footcites{foodstuffs}[rev'd][]{foodstuffs1}

\subparagraph{Lists!!}

We can have itemized (unnumbered lists):

\begin{itemize}
\item First thing
\item Second thing
\item Nested thing
\begin{itemize}
 \item Nested item 1
 \item Nested item 2
\end{itemize}
\end{itemize}

Or we can have numbered lists. With these we can tell them how
to write the numbers:

\begin{enumerate}[(a)]
\item First
\item Second item in the list
\end{enumerate}

Or with arabic numbers with internal citations. 
\begin{enumerate}[(1)]
\item An item\footcite{reekie}
\item too many items\footcite{tuhou}
\end{enumerate}

\cleardoublepage

\section{Bibliography}

Finally we want a bibliography section, and we want it to be on its own
facing page. In future it is hoped that 
bibliographies will be automatically generated. However, the rules
behind the bibliographies are rather complicated and I have not
gotten round to doing that yet. So instead we have the following.

First we tell it to forget all about what citations it has done.
\newrefsection

Then we want to make the spacing a bit nicer so that the citations are
not indented like paragraphs.
\parindent=0pt

Then we use the special \verb|\cite| command to print a bibliography
entry (not in a footnote). This can of course be done anywhere.

\subsection{Cases}

\cite{reekie}

\cite{mafart}

\cite{z}

\cite{astrazeneca}

\subsection{Legislation}
\cite{gaming}


That said there has been some progress along the line to using
biblatex commands to print out the bibliography. But it isn't 
quite working write.

\nocite{*}
\printbibliography[title={Cases},type=case,sorting=nyt]


Or in complete form:
\printbibliography
\end{document}
